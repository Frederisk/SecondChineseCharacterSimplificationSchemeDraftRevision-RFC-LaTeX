\documentclass[12pt, fontset=none, UTF8, AutoFakeBold]{ctexbook}

\usepackage[outline]{contour}

\usepackage{geometry}
    \geometry{a4paper, centering, top=2.54cm, bottom=2.54cm, left=1.27cm, right=2.54cm}

\usepackage{xeCJK}
    \setCJKmainfont[
        ItalicFont=FangSong
    ]{BabelStone Han}
    % \ctexset{section={name={,、}, number={\chinese{section}}}}

\usepackage{setspace}
    \renewcommand{\baselinestretch}{1.75}

\usepackage{fancyhdr} % page header and footer
    \pagestyle{fancy}
    % \fancyhf{}
    \fancyhead[]{}
    \renewcommand{\headrule}{} % no header rule
    \fancyfoot[C]{--~\thepage~--}

\usepackage{titlesec}
    \titleformat{\section}{\centering\Large}{\chinese{section}、}{0pt}{}

\newcommand{\raisesymbol}[1]{\raise1.3pt\hbox{#1}}

\begin{document}
\begin{titlepage}
    % \thispagestyle{empty}
    \begin{center}
        \huge
        \vspace*{3cm}
        \textbf{第二次汉字简化方案修订草案}\par
        % \vspace*{0.5cm}
        \textbf{征\ 求\ 意\ 见\ 表}\par
        % \textbf{第二次汉字简化方案修订草案}\par
        % \textbf{征求意见表}
        \vspace*{2.5cm}
        \Large
        说\qquad{}明\par
    \end{center}

    《第二次汉字简化方案(草案)》于1977年12月发表后,在全国范围征求了意见,26个省市自治区以及部队系统将征集到的意见综合整理后寄给了我们。由于这个草案制定得不够妥善,各方人士也提了不少批评意见。根据充实和加强后的中国文字改革委员会第一次全体会议决定,对这个草案进行了较大的修改,作成《修订草案》,现在再一次印发征求意见。

    原草案共收整体简化字462个,经过修订,现减为111个。其中不作简化偏旁用的简化字91个,可作简化偏旁用的简化字20个。《修订草案》保留了原草案的简化字79个,修改简化形体的27个。另外,补充简化了5个字。

    征求意见表分原字、简作、和意见三栏,后面附「说明」一栏,主要讲简化字的选取理由,以及对原草案作了哪些修改等,供参考。

    填写意见时请注意:你赞成的简化字,就在「意见」栏画个「\raisesymbol{$\bigcirc$}」,不赞成的就画个「\raisesymbol{$\times$}」。

    \begin{center}
        \Large
        \vspace*{1.5cm}
        \textit{中\ 国\ 文\ 字\ 改\ 革\ 委\ 员\ 会}\par
        \textit{1981年8月}
    \end{center}
\end{titlepage}

% \section{}

\end{document}
