\documentclass[12pt, fontset=none, UTF8, AutoFakeBold]{ctexbook}

\usepackage[outline]{contour}
% \usepackage{longtable}
\usepackage{tabularx}
\usepackage{booktabs}
\usepackage{ltablex}

\usepackage{geometry}
    \geometry{a4paper, centering, top=2.54cm, bottom=2.54cm, left=1.27cm, right=2.54cm}

\usepackage{fontspec}
    \setmainfont[ItalicFont=FangSong]{BabelStone Han}[AutoFakeBold]

\usepackage{xeCJK}
    \setCJKmainfont[
        ItalicFont=FangSong
    ]{BabelStone Han}
    \newCJKfontfamily\bse{BabelStone Erjian 1}
    \newCJKfontfamily\bsee{BabelStone Erjian 2}

    % \ctexset{section={name={,、}, number={\chinese{section}}}}

\usepackage{setspace}
    \renewcommand{\baselinestretch}{1.75}

\usepackage{fancyhdr} % page header and footer
    \pagestyle{fancy}
    % \fancyhf{}
    \fancyhead[]{}
    \renewcommand{\headrule}{} % no header rule
    \fancyfoot[C]{--~\thepage~--}

\usepackage{titlesec}
    \titleformat{\section}{\centering\Large}{\chinese{section}、}{0pt}{}

\newcommand{\raisesymbol}[1]{\raise1.3pt\hbox{#1}}

\begin{document}
\begin{titlepage}
    % \thispagestyle{empty}
    \begin{center}
        \huge
        \vspace*{3cm}
        \textbf{第二次汉字简化方案修订草案}\par
        % \vspace*{0.5cm}
        \textbf{征\ 求\ 意\ 见\ 表}\par
        % \textbf{第二次汉字简化方案修订草案}\par
        % \textbf{征求意见表}
        \vspace*{2.5cm}
        \Large
        说\qquad{}明\par
    \end{center}

    《第二次汉字简化方案(草案)》于1977年12月发表后,在全国范围征求了意见,26个省市自治区以及部队系统将征集到的意见综合整理后寄给了我们。由于这个草案制定得不够妥善,各方人士也提了不少批评意见。根据充实和加强后的中国文字改革委员会第一次全体会议决定,对这个草案进行了较大的修改,作成《修订草案》,现在再一次印发征求意见。

    原草案共收整体简化字462个,经过修订,现减为111个。其中不作简化偏旁用的简化字91个,可作简化偏旁用的简化字20个。《修订草案》保留了原草案的简化字79个,修改简化形体的27个。另外,补充简化了5个字。

    征求意见表分原字、简作、和意见三栏,后面附「说明」一栏,主要讲简化字的选取理由,以及对原草案作了哪些修改等,供参考。

    填写意见时请注意:你赞成的简化字,就在「意见」栏画个「\raisesymbol{$\bigcirc$}」,不赞成的就画个「\raisesymbol{$\times$}」。

    \begin{center}
        \Large
        \vspace*{1.5cm}
        \textit{中\ 国\ 文\ 字\ 改\ 革\ 委\ 员\ 会}\par
        \textit{1981年8月}
    \end{center}
\end{titlepage}

\section{不作简化偏旁用的简化字\\(91个)}

\begin{tabularx}{\linewidth}{|c|c|c|c|X|}
    \toprule
    编号 & 原字 & 简作 & 意见 & 说明 \\
    \midrule
    1 & 癍 & 斑 & & \quad{}用斑代替,不会引起歧义。 \\
    2 & 舨 & 板 & & \quad{}舢舨同舢板。舨是后起字,合并后不会引起歧义。 \\
    3 & 爆 & 𤆊 & & \quad{}流行较广,部队中普遍通行。 \\
    4 & 萹稨藊 & 扁 & & \quad{}萹豆、稨豆、藊豆也作扁豆。见《现代汉语词典》(萹又音bi\={a}n,萹蓄、萹竹仍用萹)。 \\
    5 & 彩 & 采 & & \\
    6 & 餐 & 歺 & & \\
    7 & 藏 & 䒙 & & \\
    8 & 撤 & 𢪃 & & \\
    9 & 澈 & 彻 & & \\
    10 & 瞅 & 𥄨 & & \\
    11 & 矗 & ⿳十且双 & & \\
    12 & 答 & 荅 & & \\
    13 & 巅癫 & ⿰⿳十且八页 & & \\
    14 & 蠹 & 𧉓 & & \\
    15 & 锻 & 煅 & & \\
    16 & 蹲 & 𧿬 & & \\
    17 & 燉 & 炖 & & \\
    18 & 孵 & 孚 & & \\
    19 & 副 & 付 & & \\
    20 & 覆 & 覄 & & \\
    21 & 赣 & ⿱夂贡 & & \\
    22 & 罐 & ⿰缶关 & & \\
    23 & 灌 & ⿰⺡关 & & \\
    24 & 鱖 & 鳜 & & \\
    25 & 薅 & 䒵 & & \\
    26 & 寰 & 环 & & \\
    27 & 毀 & ⿰圼殳 & & \\
    28 & 籍 & 笈 & & \\
    29 & 简 & 𫈉 & & \\
    30 & 疆 & ⿰弓⿱𰢴一 & & \\
    31 & 缰 & ⿰⺰⿱𰢴一 & & \\
    32 & 僵 & ⿰⺅⿱𰢴一 & & \\
    33 & 诫 & 戒 & & \\
    34 & 襟 & 衿 & & \\
    35 & 橘 & 桔 & & \\
    36 & 飓 & 巨 & & \\
    37 & 镢䦆 & 𰽤 & & \\
    38 & 蝌 & 科 & & \\
    39 & 鬎 & 瘌 & & \\
    40 & 镰 & 𰾮 & & \\
    41 & 量 & 𰅊 & & \\
    42 & 嘹 & 𠮩 & & \\
    43 & 僚 & 𠆨 & & \\
    44 & 寮 & 㝋 & & \\
    45 & 燎 & 𰝶 & & \\
    46 & 潦 & 𣱾 & & \\
    47 & 镣 & 钌 & & \\
    48 & 掳 & 虏 & & \\
    49 & 貌 & 皃 & & \\
    50 & 没 & 沒(⿰氵𠬛) 没 \bse 没 \bsee 没 & & \\
    51 & 腻 & 胒 & & \\
    52 & 漆 & 㲺 & & \\
    53 & 歧 & 岐 & & \\
    54 & 壤 & 圵 & & \\
    55 & 嚷 & 𠮵 & & \\
    56 & 赛 & 𡧳 & & \\
    57 & 砂 & 沙 & & \\
    58 & 搧煽 & 扇 & & \\
    59 & 膻 & \bsee 膻 & & \\
    60 & 鞝绱 & 上 & & \\
    61 & 杓 & 勺 & & \\
    62 & 算 & 祘 & & \\
    63 & 檀 & 枟 & & \\
    64 & 套 & 𰋛 & & \\
    65 & 腾 & 𱅑 & & \\
    66 & 滕藤 & 𣳾 & & \\
    67 & 嚏 & ⿰口⿳十冖疋 & & \\
    68 & 臀 & \bsee 臀 & & \\
    69 & 豌 & 宛 & & \\
    70 & 稀 & 希 & & \\
    71 & 鑫 & 𨥖 \bse 鑫 \bsee 鑫 & & \\
    72 & 信 & 伩 & & \\ % TODO: 信 傳統體
    73 & 癣 & 㾌 & & \\
    74 & 阎 & 闫 & & \\
    75 & 耀曜 & 𭀦 & & \\
    76 & 臆 & 肊 & & \\
    77 & 癔 & 𤴥 & & \\
    78 & 意 & 𰐺 & & \\
    79 & 翼 & 𮊾 \bse 翼 \bsee 翼  & & \\
    80 & 鹰 & ⿰应鸟 & & \\
    81 & 迎 & 迊 & & \\
    82 & 赢 & ⿱亡⿵几贝 & & \\
    83 & 臃 & 𦙸 & & \\
    84 & 粤 & ⿳丿𭁨丂 & & \\
    85 & 账 & 帐 & & \\
    86 & 辙 & 𰹹 & & \\
    87 & 整 & 𰋞 & & \\
    88 & 谘 & 咨 & & \\
    89 & 籽 & 子 & & \\
    90 & 鬃 & 骔 & & \\
    91 & 纂 & ⿱𰩬糸 & & \\
    \bottomrule
\end{tabularx}



\end{document}
